\documentclass{ctexart}

\usepackage{appendix}
\usepackage{listings}% 插入代码
\usepackage{xcolor} 
\usepackage{graphicx}% 插入表格/图片
\usepackage{booktabs} % 绘制表格
\usepackage{caption} % 标题
\usepackage{geometry}
\usepackage{array}
\usepackage{amsmath}
\usepackage{subfigure} % 插入图片
\usepackage{longtable}
\usepackage{abstract}% 摘要
\pagestyle{plain} % 页眉消失
\usepackage{setspace}
\usepackage{multirow}% 表格
\usepackage{diagbox}
\usepackage{enumerate}% 序号
\usepackage{float}% 固定图片或表格的位置
\usepackage{gensymb}
\usepackage{microtype}
\usepackage{amsmath}
\usepackage{center}

\geometry{a4paper,left=2.5cm,right=2.5cm,top=2cm,bottom=2cm}% 页边距
\lstset{
    numbers=left, % 设置行号位置
    numberstyle=\tiny, % 设置行号大小
    keywordstyle=\color{blue}, % 设置关键字颜色
    commentstyle=\color[cmyk]{1,0,1,0}, % 设置注释颜色
    escapeinside=``, % 逃逸字符(1左面的键),用于显示中文
    breaklines, % 自动折行
    extendedchars=false, % 解决代码跨页时,章节标题,页眉等汉字不显示的问题
    xleftmargin=1em,xrightmargin=1em, aboveskip=1em, % 设置边距
    tabsize=4, % 设置tab空格数
    showspaces=false % 不显示空格
}

\title{论文设计}
\date{}
\author{}

\begin{document}
\maketitle
\renewcommand{\abstractname}{\Large\textbf{摘要}\\} % 使用 \huge 调整字体大小
\vspace{-4em} % 调整标题上间距
\begin{abstract}
\normalsize
本文针对问题,建立了等多种模型,解决了问题。

针对问题一,

针对问题二,

针对问题三,

\documentclass{article}

\begin{document}

\item{引力方程和离心力平衡}
\begin{center}
\[G \frac{M m}{\rho^2} = m \frac{v^2}{\rho}\]
\end{center}

\item{近地点和远地点高度}
\begin{center}
\[h_p = 15 \text{ km}\]
\[h_a = 100 \text{ km}\]
\end{center}

\item{地心距离}
\begin{center}
\[\rho_p = r + h_p\]
\[\rho_a = r + h_a\]
\[r = 1737.013 \text{ km}\]
\end{center}

\item{半长轴}
\begin{center}
\[a = \frac{\rho_a + \rho_p}{2}\]
\[a = \frac{(r + h_a) + (r + h_p)}{2}\]
\end{center}

\item{半焦距}
\begin{center}
\[c = a - (h_p + r)\]
\end{center}

\item{速度}
\begin{center}
\[v = \sqrt{\frac{G M}{r}}\]
\end{center}

\item{半短轴}
\begin{center}
\[b^2 = a^2 - c^2\]
\end{center}

\item{半径}
\begin{center}
\[\rho = \frac{b^2}{a}\]
\end{center}

\item{数值代入计算}

\item{地心距离}
\begin{center}
\[P_p = R + h_p = 1737.013 + 15 = 1752.013 \text{ km}\]
\[P_a = R + h_a = 1737.013 + 100 = 1837.013 \text{ km}\]
\end{center}

\item{半长轴}
\begin{center}
\[a = \frac{P_a + P_p}{2} = \frac{1837.013 + 1752.013}{2} = 1794.513 \text{ km}\]
\end{center}

\item{半焦距}
\begin{center}
\[c = a - (h_p + R) = 1794.513 - (15 + 1737.013) = 42.5 \text{ km}\]
\end{center}

\item{半短轴}
\begin{center}
\[b^2 = a^2 - c^2 = 1794.513^2 - 42.5^2\]
\[b = \sqrt{1794.513^2 - 42.5^2}\]
\end{center}

\item{半径}
\begin{center}
\[r = \frac{b^2}{a}\]
\end{center}

\item{速度}
\begin{center}
\[v = \sqrt{\frac{G M}{r}}\]
\end{center}

\end{document}


\textbf{关键字}:
\end{abstract}
\newpage



% 重新设置页面边距
    \newgeometry{a4paper,left=3.18cm,right=3.18cm,top=2.54cm,bottom=2.54cm}
	\section{问题背景与重述}
	\subsection{问题背景}
    在
    \subsection{问题表述}
    \begin{enumerate}[(1)]
        \item 问题一:
        \item 问题二:
        \item 问题三:
    \end{enumerate}

    \section{问题分析}
    \subsection{问题一分析}
    对于问题一
    \subsection{问题二分析}
    首先
    \subsection{问题三分析}
    对于该问题
    \section{模型假设}
    \begin{enumerate}[(1)]
        \item 
        \item 
        \item 不考虑
        \item 假设
    \end{enumerate}

    \section{符号说明}
\begin{center}
    \setlength{\tabcolsep}{9mm}{
        \begin{tabular}{ccc}
            \specialrule{1.2pt}{0pt}{0pt} % 设置顶部粗线
            \textbf{符号} & \textbf{意义} & \textbf{单位}\\
            \midrule  % 设置中间横线
            \textnormal{} & \textnormal{} & \textnormal{}\\
            \textbf{符号} & \textbf{意义} & \textbf{单位}\\
            \textnormal{} & \textnormal{} & \textnormal{}\\
            \textbf{符号} & \textbf{意义} & \textbf{单位}\\

            \specialrule{1.2pt}{0pt}{0pt} % 设置底部粗线
        \end{tabular}
    }
\end{center}

    \section{模型建立与求解}
    \subsection{问题一模型的建立与求解}
    \subsubsection{问题一模型的建立}
针对这一问题建立了多种优化模型,
    
    \subsubsection{数据预处理}
	
考虑到

\subsubsection{计算}
\subsubsection{结果的检验}
\subsubsection{问题的结论}
    \subsection{问题二模型的建立与求解}
\subsubsection{数据预处理}
\subsubsection{模型建立}
\subsubsection{模型建立的数学思想}
\subsubsection{模型建立的准备}
\subsubsection{模型的求解}
\subsubsection{模型求解的数学原理}
\subsubsection{模型求解的准备}
\subsubsection{模型求解的过程}
\subsubsection{模型求解的结果}
\subsubsection{结果的检验}
\subsubsection{问题的结论}
\subsubsection{模型检验与分析}
\subsubsection{误差分析}
\subsubsection{灵敏度分析}
\subsubsection{稳定性分析}
\subsubsection{小结}
    \subsection{问题三模型的建立与求解}
\subsubsection{数据预处理}
\subsubsection{模型建立}
\subsubsection{模型建立的数学思想}
\subsubsection{模型建立的准备}
\subsubsection{模型的求解}
\subsubsection{模型求解的数学原理}
\subsubsection{模型求解的准备}
\subsubsection{模型求解的过程}
\subsubsection{模型求解的结果}
\subsubsection{结果的检验}
\subsubsection{问题的结论}
\subsubsection{模型检验与分析}
\subsubsection{误差分析}
\subsubsection{灵敏度分析}
\subsubsection{稳定性分析}
\subsubsection{小结}
    \subsection{问题四模型的建立与求解}
\subsubsection{数据预处理}
\subsubsection{模型建立}
\subsubsection{模型建立的数学思想}
\subsubsection{模型建立的准备}
\subsubsection{模型的求解}
\subsubsection{模型求解的数学原理}
\subsubsection{模型求解的准备}
\subsubsection{模型求解的过程}
\subsubsection{模型求解的结果}
\subsubsection{结果的检验}
\subsubsection{问题的结论}
\subsubsection{模型检验与分析}
\subsubsection{误差分析}
\subsubsection{灵敏度分析}
\subsubsection{稳定性分析}
\subsubsection{小结}
    \begin{thebibliography}{9} % 参考文献
		\bibitem{bib:8}何晓群.多元统计分析.北京:中国人民大学出版社,2012.
		\bibitem{bib:9}徐维超. 相关系数研究综述[J]. 广东工业大学学报,2012,29(3):12-17.
    \end{thebibliography}

    \newpage
    \section{附录}
    %插入代码内容
\begin{lstlisting}
		\end{lstlisting}
\end{document}       